% Activate the following line by filling in the right side. If for example the name of the root file is Main.tex, write
% "...root = Main.tex" if the chapter file is in the same directory, and "...root = ../Main.tex" if the chapter is in a subdirectory.
 
%!TEX root =  testMain.tex

\chapter[Abstract]{Abstract}

There are probabilistic methods for reasoning with evidence in court cases. Bayesian Networks have been suggested as one of these methods, but we do not know how suitable they would be in the domain of law, which is not traditionally a statistical field. This work investigates Bayesian Networks for criminal cases by creating multi-agent simulations of crimes, collecting frequency data on these simulations, using the K2 algorithm to generate Bayesian Networks from this data, and then evaluating the generated Bayesian Networks on structural, performance and human factor criteria. This is done for 3 simple simulations: one non-spatial simulation of a murder, one spatial simulation of a home-robbery with evidence, and one spatial simulation of a street robbery in a non-trivial environment.

We find that the Bayesian Networks are generally accurate and perform well even if we do not know the exact probabilities as generated by the simulation. However, it is implausible that human modellers could replicate the cpt's of the networks. Apart from that, private knowledge, conditioning on implicit factors and the effects of non-ideal operationalisation are factors that need to be solved before Bayesian Networks can be responsibly applied in court.