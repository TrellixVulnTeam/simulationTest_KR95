%!TEX root =  testMain.tex

\chapter[State of the Art]{State of the Art}

We want to create grounded Bayesian Networks for reasoning with evidence. This can be done by creating multi-agent simulations, and building Bayesian Networks based on the events in these simulations. In this section, I explain the relevant background on reasoning with evidence, Bayesian Networks, and multi-agent simulations.

\section{Reasoning with evidence}
We consider three main directions in the field of representing reasoning with evidence within law \citep{Verheij2015}. The first direction is via argumentation approaches, where hypotheses and evidence are represented as propositions that attack or support each other. The second direction is via scenario approaches, where more-or-less coherent hypotheses are combined into stories \citep{Pennington1993}, which are supported with evidence. The third direction is via probabilistic approaches, where hypotheses and evidence are assigned probabilities, and the relation between hypothesis and evidence is represented with conditional probability. There are also be hybrids that combine or synthesise aspects of each approach, see, for example: \citep{Bex2010} and \citep{Timmer2016}. This project does not consider the argumentative approach, and is mainly based on the work of Vlek \citep{Vlek2015}, \citep{Vlek2016} and Fenton \citep{Fenton2012}, \citep{Fenton2019}. The networks in these papers describe whole criminal situations (scenarios), but use Bayesian Networks and hence have a probabilistic component.


\section{Bayes and Bayesian Networks}

We have already seen the power of Bayes's law in the introduction. It tells us how much we have to change our belief in a hypothesis once we find a piece of evidence. To do this, we need to have a lot of information (the likelihood ratio and the prior). 

Bayesian reasoning, without Bayesian Networks, have been used by Dahlman in `event trees', specifying different branches of combinations and their probabilities \citep{dahlman2020}.

However, just using Bayes law is very difficult, because sometimes we want to condition on multiple pieces of evidence. Doing all the calculations is tedious, hence we use the computational tools called Bayesian Networks. 


Bayesian Networks can represent aspects of a criminal case or can attempt a scenario-like hybrid and represent the entire case - modelling actual crime cases \citep{Kadane1996},  simonshaven, cases from fiction \citep{Fenton2012} or fictionalized crime cases \citep{vanLeeuwen2019}. In situations where the network represents aspects, the nets model DNA or blood-spatter evidence, for methods see \citep{Meester2021}. 

Bayesian Networks can be built by hand, by experts or academics, in (proprietary) software like AgenaRisk or Hugin. They can also be built by hand in PyAgrum \citep{pyagrum2020}, a free Python software package, which does not have a GUI but has everything else. In this project, PyAgrum was used. Alternatively, Bayesian Networks can be automatically constructed from large datasets - PyAgrum also offers the opportunity for that - (documentation). Automated Bayesian Network building is not plausible in the legal-evidence domain, because the data that we need is notoriously sparse - we have information about the number of crimes, from police departments, but for the subaspects for each scenario, it is hard to find frequencies (reference class problem, and others). 

%However, since we're using simulations to investigate the Bayesian Approach, we can generate a near infinite amount of information, and then we use this information to build the network, such as we might use data on health markers to predict kidney failure (medical domain - pretty sure this is the standard example). So automated building tools come in handy. In this project I only used the K2 algorithm, because you can add temporal information to it.

\subsection{Evaluating Bayesian Networks}

Bayesian Networks are a promising tool, but there are a lot of open questions:

\begin{enumerate}
\item The use of Bayesian Networks is not straightforward, neither for the builder or for the interpreter. From \citet{deKoeijer2020}: it is complex, time-consuming, hard to explain, and, the `repeatability [...] leaves much to be desired. Node definitions and model structures are often directed by personal habits, resulting in different models for the same problem, depending on the expert'.
\item The granularity of the network - how do we know which nodes hypotheses and pieces of evidence to include? Ideally we would model as detailed as possible, but as we increase the number of nodes, we increase the complexity, which increases the probability of mistakes, and the time-spend on the network.
\item The links: how do we know which events depend on each other?
\item The numbers - there's not just subjectivity in selecting the nodes, and drawing the links, but the probabilities that we have to fill in into the cpt themselves are the most obvious stumbling block. We can identify that there has to be some sort of correlation between `smoke' and `fire', but how to express this in numbers? Fenton argues that we're just making something explicit that we would otherwise have left implicit. But if we `explicitivize' wrongly? How robust is the network against imprecise or wrong frequencies? And if we use frequencies (or subjective probabilities based on frequencies), how do we decide what set of events is included when we start to count (this is the problem of the reference class, see \citep{Allan2007}, \citep{colyvan2001}). Probabilities can be pure frequencies, or can be subjectively elicited from experts \citep{renooij2001}, \citep{Druzdzel2000}.
\end{enumerate}


In this project, I can not address all of these problems, because it's too much. My main focus will be on problem 4. As we mentioned before, one of the problems that can plague Bayesian Network creation is the lack of well-defined and plentiful data. But what if we would have this data about criminal cases? Then we could test our methods for Bayesian Networks without worrying about the subjectivity and lack of frequency of the numbers in the cpt. If we can create a grounded `data-generating' environments for criminal cases, we can test our Bayesian Networks against them.

\section{Agent simulations}
We can create such a data-generating environment for criminal cases by the use of multi-agent simulations. In a multi-agent simulation, agents observe and interact with their environment. The environment and the agent behaviour is fully controlled by its programming and all randomness can be accounted for. This means that we know exactly with which frequencies fully-specified events occur. Running the simulation multiple times generates a lot of data, which will serve as input to automatically build a Bayesian Network from data. This means we use a multi-agent simulations as a ground for our theory-testing. In this project, the simulations will be programmed in Python using the MESA framework  \citep{mesa2020}.


Multi-agent simulations have been used to investigate the criminal domain and to test out sociological theories. By creating spatially explicit simulations, the complex interactions of agents with their locations can be represented better than in traditional models - these agent-based models can model criminal hotspots \citep{Gerritsen2008}, theories of behaviour \citep{Gerritsen2015}, and police strategies in urban crime \citep{Zhu2021}. Weaknesses identified with multi-agent models for criminological research are insufficiently complex models, unclarity on the effect of temporal resolution, and lack of a systemic approach \citep{Zhu2021}.



 \section{Conclusion}

The two main ideas are Bayesian reasoning, specifically as implemented in Bayesian Networks, and computational simulations of agents. The simulation is supposed to be the grounding for the Bayesian Networks.

But why do we want to use Bayesian Networks in the first place? In the best possible case, a Bayesian Network would help us to make the correct reasoning steps, telling us exactly how to weigh each piece of evidence in the grand scale of the simulated crime. This is a normative approach - it's telling us how to reason, but it is not (and not meant to be) an empirical approach. As far as I know, Bayesian networks are not a reflection of `how people actually reason' - eg if you open up our minds, you won't see Bayesian Networks in there. One of the problems for normative models in decision making \citep{colyvan2013}, is that we can only test how well our normative models work, if we already know what a good outcome would be. In a sense, we're doing experimental model testing for normative Bayesian Networks in this project.



% This is not necessarily a bad thing - we put our (betting) money where our mouth is and assign numerical precise probabilities to situations that we preciously only had vague intuitions for. However, this brings about the veil of objectivity. By giving a probability to your intuitions, you have made your intuitions more precise and you can now reason with them, update on evidence using Bayes Law, and everything's great. In some domains, this is obviously okay. If you want to bet cents on world events and walk that fine line between calibration and discrimination for fun and profit, that's no problem.  After all, there are incentives to abstain from the unclear, the stuff that might not have a specified answer, the vague. But when we are talking about using evidence in law, we are talking about exactly that domain - we're not making predictions about the price of oil in 6 months, or the outcome of the French election, with clear outcomes, clear procedures for measurement. Instead, we're trying to make predictions about crimes and crime scenarios, which are a lot vaguer, and strangely unobservable at times - things like motives, or behaviours that happen under specific circumstances, interlocking stuff with complex dependencies. We all have intuitions about evidence strengths in vague situations, but they are more difficult to make precise than the traditional `forecasting' events. So trying to assign probabilities without a clear method of calibration, makes that they will be imprecise.

