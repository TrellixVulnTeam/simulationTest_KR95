%!TEX root =  mainMastersProject.tex

\chapter[State of the Art]{State of the Art}

\section{The problem with evidence}

When we find evidence for a hypothesis that we have held in the back of our minds, we then find the hypothesis more likely. 
This is the basic idea behind reasoning with evidence. In a constellation of hypotheses and pieces of evidence, we want
to construct a network that will lead us to believe as many true hypotheses as possible, given the evidence that we have.

However, evidence itself is elusive, and it's connection to hypotheses is as well - how can we be sure that some evidence supports
some hypothesis, and even if we know that it does, how can we express how strong the piece of evidence is? Some evidence
is very weak, and only after a tedious process of ruling out other factors and careful investigation and collection of other pieces
of evidence, we can come to a conclusion about a hypothesis. On the other hand, some evidence is so strong that it leaves no room for doubt.

We all have intuitions about evidence strength - but can we make these intuitions precise? Additionally, can we manage the complex
realities of weak evidence for many different hypothesis? These are questions that have guided this state of the art section.

In this section I will briefly discuss methods for reasoning with evidence, then transition to probabilistic reasoning with evidence,
Bayesian Networks. I will discuss some problems with Bayesian Networks.

\section{Reasoning with evidence}
Argumentation graphs and their semantics.
Scenario theory.
Bayes Law.
Hybrid Approach.

\section{Bayes!}
Probabilistic reasoning (Dahlman, simple).
Bayesian Networks
Fenton and Vlek - combining scenarios, idioms in Bayesian Networks.

\section{Problems with the Bayesian Approach}
There are many problems with the Bayesian Approach.

The most obvious is the problem of the numbers: where do we get them? We can get some of them from statistics, but we need to many numbers that we have to make some numbers up. This is not necessarily a bad thing - we put our (betting) money where our mouth is and assign numerical precise probabilities to situations that we preciously only had vague intuitions for. However, this brings about the veil of objectivity. By giving a probability to your intuitions, you have made your intuitions more precise and you can now reason with them, update on evidence using Bayes Law, and everything's great. In some domains, this is obviously okay. If you want to bet cents on world events and walk that fine line between calibration and discrimination for fun and profit, that's no problem.  After all, there are incentives to abstain from the unclear, the stuff that might not have a specified answer, the vague. But when we are talking about using evidence in law, we are talking about exactly that domain - we're not making predictions about the price of oil in 6 months, or the outcome of the French election, with clear outcomes, clear procedures for measurement. Instead, we're trying to make predictions about crimes and crime scenarios, which are a lot vaguer, and strangely unobservable at times - things like motives, or behaviours that happen under specific circumstances, interlocking stuff with complex dependencies. We all have intuitions about evidence strengths in vague situations, but they are more difficult to make precise than the traditional `forecasting' events. So trying to assign probabilities without a clear method of calibration, makes that they will be imprecise.

Level of granularity.

Independence relations. Selecting events.

Technical problems with Bayesian Networks. Precision. Evaluation. Sensitivity Analysis.


 



